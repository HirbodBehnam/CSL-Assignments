\smalltitle{سوال 2}

\begin{enumerate}[leftmargin=0cm,itemindent=.5cm,label=\alph*.]
    \item از آنجا که حجم این حافظه‌ی
    cache
    بیشتر است پس
    \lr{miss rate}
    کمتر است چرا که احتمال بیشتری دارد که دیتای مورد نظر در
    cache
    وجود داشته باشد.
    از طرفی حجم خانه‌هایی که باید چک کنیم که در آن دیتا وجود دارد یا نه بیشتر می‌شود و در نتیجه
    \lr{hit time}
    بیشتر می‌شود.
    \item $AMTA_a = 0.96 + 0.043 \times 70 = 3.97$, $AMTA_b = 1.06 + 0.034 \times 70 = 3.44$
    این نتیجه مشکلی ندارد. چرا که در ظاهر هر چه قدر حجم کش بیشتر باشد دسترسی کمتری به رم لازم است.
    پس در نتیجه
    AMTA
    پایین می‌آید.
    دقت کنید که ساخت کش گران است و به همین دلیل کش‌ها حجم کمی دارند.
    \item $AMTA_c = 1.06 + 0.034 \times (4 + 0.95 \times 70) = 3.457$, $AMTA_d = 1.06 + 0.034 \times (11 + 0.70 \times 70) = 3.1$
    \item c
    آنرا بدتر کرده است و
    d
    آنرا بهتر. به نظر من دلیل آن این است که ظرفیت
    c
    خیلی با ظرفیت
    b
    فرقی ندارد در نتیجه
    \lr{miss ratio}
    خیلی بالاتری دارد.
\end{enumerate}