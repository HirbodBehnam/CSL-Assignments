\smalltitle{سوال 1}

\begin{enumerate}[leftmargin=0cm,itemindent=.5cm,label=\alph*.]
    \item باید به این موضوع فکر کنیم که بعد از اجرای مرحله‌ی آخر یک دستور‌العمل، 3 میلی‌ثانیه بعد دستور العمل بعدی تمام می‌شود. پس
    $\frac{15}{3}=5$
    برابر سریع‌تر است.
    \item برای این سوال بدین صورت فکر می‌کنیم که وقتی که یک دستور
    stall
    دار وارد
    pipeline
    می‌شود اجرای هر دستور‌العمل به صورت میانگین 6 میلی‌ثانیه طول می‌کشد. چرا که همه‌ی دستور‌العمل‌ها باید یک واحد صبر کنند.
    حال این موضوع را در نظر داشته باشید که وقتی یک دستور
    stall
    اجرا می‌شود، از آنجا که
    pipeline
    5 مرحله است،
    به صورت میانگین یک دستور
    stall
    دیگر نیز وارد می‌شود. پس به صورت میانگین همه‌ی دستور‌العمل‌های ما 6 میلی‌ثانیه طول می‌کشند که سرعت آن
    $\frac{15}{6}$
    برابر می‌شود.
    \item دقت کنید که مانند قسمت اول بعد از هر \lr{3.1} میلی‌ثانیه یک دستور انجام می‌شود. پس
    $\frac{15}{3.1} = 4.8$
    سریع‌تر از ماشین A است.
    این بدین معنا است که ماشین
    B
    سریع‌تر از
    C
    است.
    \item $A: 15 \times 10 = 150$, $B: 3 \times (10 + 5 - 1) = 42$, $C: 3.1 \times (10 + 4 - 1) = 40.3$
    این تفاوت به خاطر این است که تعداد دستورالعمل‌هایی که در
    pipeline
    می‌روند کم هستند.
    برای اینکه نقطه تقاطع را پیدا کنیم باید معادله‌ی
    $3 \times (x + 5 - 1) = 3.1 \times (x + 4 - 1)$
    را حل کنیم که جواب آن 27 است.
    پس از 27 دستور به بعد سرعت
    B
    بیشتر از
    C
    می‌شود.
    \item $A: 15 \times 100 = 1500$, $B: 3 \times (100 + 5 - 1) = 312$, $C: 3.1 \times (100 + 4 - 1) = 319.3$
    \item $T = t \times (N + p - 1)$
    که در اینجا
    $t$
    زمان اجرا شدن هر مرحله‌ی
    pipeline
    است و
    $T$
    زمان نهایی است که برای اجرای
    $N$
    دستور صرف می‌شود.
    حال فرض کنید که مانند ماشین
    A
    هر دستور بدون
    \lr{pipeline}
    در
    $tp$
    زمان اجرا می‌شود.
    پس اجرای
    $N$
    دستور‌العمل‌
    $Ntp$
    زمان می‌برد.
    پس زمان تسریع
    $\frac{Ntp}{t \times (N + p - 1)} = \frac{Np}{N + p -1}$
    برابر می‌شود.
    پس زمانی سرعت حداکثر می‌شود که تعداد عملیات به بی‌نهایت میل کند.
    همچنین دقت کنید که در همه‌ی دستورات فرض کرده بودیم که
    \lr{stall}
    وجود ندارد. پس این هم یک نکته در تسریع سرعت است.
    \item در این ماشین دقت کنید که باید میانگین اجرا شدن هر دستور بعد از دستور قبلی را 3 میلی‌ثانیه بگیریم.
    چرا که بالاخره دستوراتی که 4.2 میلی‌ثانیه اجرا شدنشان طول می‌کشد، حتما باید 6.0 میلی‌ثانیه صبر کنند که
    pipeline
    جلو رود. به همین منظور به صورت متوسط هر دستور 3 میلی‌ثانیه اجرا شدنش طول می‌کشد پس سرعت این ماشین
    با ماشین
    B
    برابر است.
    \item دقت کنید که اجرا شدن یک دستور به صورت کامل
    $3 \times 6 = 18 \text{ms}$
    زمان می‌برد. یعنی اینکه اگر قرار بود هر دستور را جدا اجرا کنیم 18 میلی‌ثانیه زمان می‌برد. ولی در ماشین
    B
    هر دستور جداگانه 15 میلی‌ثانیه زمان می‌برد.
    در ماشین
    D
    عملا سرعت اجرای هر دستور بیشتر است (کند تر اجرا می‌شود) ولی موازی سازی بهتری دارد. پس بله پاسخ سوال قبل
    اوکی است و مشکلی ندارد.
\end{enumerate}