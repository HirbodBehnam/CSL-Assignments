\smalltitle{سوال 5}
\begin{enumerate}[leftmargin=0cm,itemindent=.5cm,label=\alph*.]
    \item $T = 60 \times 0.8 + (250 - 60) = 238$ در نتیجه $\frac{250-238}{250} \times 100 = 4.8\%$ سریع‌تر می‌شود.
    \item $\frac{250 - T}{250} \times 100 = 8 \implies T = 230 = 40x + (250 - 40) \implies x = 0.5$ در نتیجه 50 درصد سرعت پرش را زیاد کرده‌ایم.
    \item $\frac{250 - (60x + (250 - 60))}{250} = 0.25 \implies 60 - 60x = 62.5 \implies x < 0$ پس امکان ندارد.
    \item فرض کنید
    $n$
    تعداد دستور عدد صحیح و
    $\frac{n}{2}$
    دستور عدد اعشاری داریم.
    سرعت اجرای هر دستور عدد صحیح برابر
    $\frac{250-60-60-40}{n} = \frac{90}{n} \text{s}$
    و سرعت هر دستور عدد اعشاری برابر
    $\frac{60}{\frac{n}{2}} = \frac{120}{n} \text{s}$
    است.
    حال بعد از تغییر
    $n + \frac{n}{4} = \frac{5n}{4}$
    دستور عدد صحیح داریم و
    $\frac{n}{4}$
    دستور عدد اعشاری داریم.
    حال با این اعداد مدت زمان اجرا را بدست می‌آوریم:
    $T = \frac{5n}{4} \times \frac{90}{n} + \frac{n}{4} \times \frac{120}{n} = 142.5 \text{s}$
    پس برای مقدار زمان اجرای کاهش یافته داریم:
    $\frac{150-142.5}{250} \times 100 = 3\%$
\end{enumerate}