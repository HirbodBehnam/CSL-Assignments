\smalltitle{سوال 4}

\noindent
\lr{SSE}
یا
\lr{Streaming SIMD Extensions}
یک مجموعه دستورات است که به ما اجازه می‌دهد کار‌های برداری را با سرعت بیشتری به کمک
\lr{SIMD}
انجام دهیم.
زمانی که این سری
\lr{instruction set}
به سی‌پی‌یو‌ها اضافه شدند چندین رجیستر
128
بیتی نیز به پردازنده‌ها اضافه شد که این سری دستورالعمل‌ها بر روی این رجیستر‌ها عمل می‌کنند.
به صورت کلی نسخه‌ی اول
\lr{SSE}
توانایی انجام عملیات بر روی یک (یا چند) بردار در
$\mathbb{R}^4$
که شامل 4
float
32 بیتی
است را دارد.
به عنوان مثال این چهار دستور دو بردار
$v1$ و $v2$
را با هم جمع می‌کند و در
$v3$
می‌ریزد:
\codesample{codes/sse.asm}
\noindent
در ادامه نسخه‌های 2 تا 4
\lr{SSE}
نیز معرفی شدند که توانایی‌هایی همچون انجام عملیات بر روی دو بردار در
$\mathbb{R}^2$
حاوی دو عدد
double
64 بیتی،
ضرب داخلی و
\dots
را داشتند.

\noindent
در ادامه‌ی توسعه‌ی
\lr{SSE}،
سری مجموعه ‌دستوراتی به نام
\lr{AVX} یا \lr{Advanced Vector Extensions}
نیز معرفی شدند که رجیستر‌های 256 بیتی به پردازنده‌ها اضافه کردند.
به کمک این دستورات می‌توان عملیات برداری را بر روی
8 float
یا
4 double
انجام داد.
همچنین در این سری دستورالعمل‌ها برنامه‌نویس می‌توانست نتیجه‌ی دستورالعمل‌ها را در رجیستر دیگری بریزد.
همچنین در نسخه‌ی آخر
\lr{AVX-512}
رجیستر‌های 512 بیتی اضافه شدند.

\noindent
یکی از کاربرد‌های این سری
\lr{instruction}
در توسعه‌ی بازی است. چرا که بردار و فضای سه بعدی استفاده‌ی فراوان در بازی‌ها دارد و انجام عملیات برداری
به صورت سریع باعث بهبود سرعت بازی می‌شود.
یکی دیگر از کاربرد‌های این
\lr{instruction}ها
در انکود فیلم با کدک‌های
\lr{h264}, \lr{h265} و \lr{vp9}
است.
همچنین در پیاده سازی یک سری از الگوریتم‌های رمزنگاری مثل
\lr{Curve25519}
می‌توان از دستورات
\lr{AVX}
برای زیادتر کردن سرعت استفاده کرد.