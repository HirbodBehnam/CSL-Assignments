\smalltitle{سوال 2}\\
\emph{mult}:
از آنجا که این دستور مقادیر هر دو رجیستر
$\$s0$
,
$\$s1$
 را به عنوان ورودی می‌گیرد، از نوع
\lr{Register Addressing}
است.
دقت کنید که جواب
\emph{mult}
در
$hi$
و
$lo$
می‌رود که آنها
\lr{Implicit Addressing}
هستند.
\\
\emph{mflo}:
در انجا
$\$t4$
به صورت
\lr{Register Addressing}
هستند ولی مبدأ جایی که عدد می‌آید از رجیستر
$lo$
است که
\lr{Implicit Addressing}
است.
\\
\emph{lw}:
مشخص است که
$\$s2$
\lr{Register Addressing}
است. ولی
از انجا که در قسمت دوم دستور یک
offset
را نیز علاوه بر یک رجیستر دریافت می‌کند که همان
base
آدرس ما است، این 
operand
از نوع
\lr{Base Addressing}
است.
\\
\emph{addi}:
$\$s3$ و $\$t6$
از نوع
\lr{Register Addressing}
هستند.
از آنجا که یک عدد ثابت داریم در این قسمت از نوع
\lr{Immediate Addressing}
است.
\\
\emph{jr}:
از این دستور
\lr{Register Addressing}
است چرا که آدرسی که می‌خواهیم به آن برویم در
$\$ra$
است.
\\
\emph{jal}:
دقت کنید که عملا این دستور
\lr{psuedo memory direct}
است. چرا که ما نمی‌توانیم از هر جای برنامه به هر جا بپریم به خاطر محدودیت‌های
MIPS.