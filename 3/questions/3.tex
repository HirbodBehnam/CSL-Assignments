\\
\linebreak
\smalltitle{سوال 3}
\makeatletter
\newcommand{\xslalph}[1]{\expandafter\@xslalph\csname c@#1\endcsname}
\newcommand{\@xslalph}[1]{%
    \ifcase#1\or a\or b\or c\or d\or e\or f\or g\or h\or i%
    \or j\or k\or l\or m\or n\or o\or p\or q\or r\or s\or%
    \or t\or u\or v\or w\or x\or y\or z%
    \else\@ctrerr\fi%
}
\AddEnumerateCounter{\xslalph}{\@xslalph}{m}
\makeatother
\begin{latin}
\begin{enumerate}[label=\xslalph*)]
    \item $R1 \times 100 = 991 \times 100 = 991000$
    \item $R1 + R2 = 991 + 713 = 1704$
    \item $R1 - M[FGH] = R1 - EDE = 991 - 171 = 820$
    \item $R1 \times M[100] = 991 \times 200 = 198200$
    \item $R1 - M[M[102]] = R1 - M[ABC] = R1 - CHA = 991 - 139 = 852$
    \item $R1 + M[100 + 2] = R1 + ABC = 991 + 991 = 1982$
    \item $R1 - M[101] = R1 - FFH = 991 - 333 = 658$
    \item $R1 \times M[101 - 1] = R1 \times 200 = 991 \times 200 = 198200$
\end{enumerate}
\end{latin}
دقت کنید در قسمت
g
بعد از اتمام عملیات به
$R6$
یکی زیاد می‌شود.