\\
\linebreak
\smalltitle{سوال 4}\\
\smalltitle{قسمت الف}\\
در ابتدا تعداد کل
\lr{opcode}ها
را حساب می‌کنیم.
$22 + 20 + 10 + 5 = 57$
حال حداقل تعداد بیتی را می‌خواهیم که بشود با آن 57 را نشان داد.
این عدد برابر
$2^6=64$
است.
پس برای قسمت
opcode
نیاز به 6 بیت داریم.
دقیقا با همین منطق حداقل تعداد بیت برای آدرس‌دهی رجیستر‌ها را بدست می‌آوریم:
$\log_2 = 128 = 7$
پس به 7 بیت نیاز داریم (حداقل).

\noindent
حال برای هر نوع از دستور‌ها بیت‌های مورد نیاز را حساب می‌کنیم؛
در دستور نوع
A
حداقل به
$6 + 7 + 7 + 7 = 27$
بیت نیاز داریم. از آنجا که تعداد بیت‌ها باید مضرب 8 باشد باید 32 بیت به طول دستورات اختصاص دهیم.
دقت کنید که این تعداد بیت برای همه‌ی نوع‌های دستوراتمان یکسان است.
برای دستور
B
نیز به همین ترتیب است و 
$6 + 7 + 7 = 20$
بیت مفید استفاده می‌شود.
در نوع دستور
C
مجبوریم که
$6 + 7 = 13$
بیت را به
opcode و register
اختصاص دهیم. پس 
$32 - 13 = 19$
بیت برای آدرس مموری باقی می‌ماند.
در دستورات نوع
D
نیز مانند بالا فقط 6 بیت باید به
opcode
اختصاص می‌دهیم. در این حال
$32 - 6 = 26$
بیت برای آدرس حافظه باقی می‌ماند.

\smalltitle{قسمت ب}\\
دقت کنید که همچنان برای هر رجیستر به 7 بیت نیاز داریم.
حال دقت کنید که از آنجا که در کل 4 نوع دستور و
mode
داریم نیاز به دو بیت هم برای ذخیره سازی
mode
در هر حالت داریم.
حال دقیقا مثل قبل اولین عدد توان دو بزرگتر مساوی
$n$
را پیدا می‌کنیم که در آن
$n$
تعداد
\lr{opcode}ها
است.
پس برای دستورات نوع اول
$2 + 5 + 7 + 7 + 7 = 28$
بیت نیاز داریم که این موضوع نشان می‌دهد که باز هم برای ذخیره سازی دستورات به 32 بیت نیاز داریم.
حال برای دستور
B
داریم:
$2 + 5 + 7 + 7 = 21$
حال دقت کنید که برای دستور
C
فقط 10 نوع
opcode
داریم. پس برای ذخیره سازی این
opcode
این دستور به 4 بیت نیاز داریم.
پس
$32 - (2 + 4 + 7) = 19$
بیت برای آدرس حافظه باقی می‌ماند.
برای دستورات نوع
D
از آنجا که فقط 5 نوع
opcode
داریم فقط به 3 بیت نیاز داریم.
پس
$32 - (2 + 3) = 27$
بیت برای آدرس مموری باقی می‌ماند.