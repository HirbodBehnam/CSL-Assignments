\smalltitle{سوال 1}\\
\smalltitle{الف)}
\codesample{codes/1-1.asm}
\smalltitle{ب)}
\codesample{codes/1-2.asm}
\smalltitle{ج)}
\codesample{codes/1-3.asm}
\smalltitle{د)}
در ابتدا عبارت را به صورت
postfix
می‌نویسیم.
برای این کار عبارت را به صورت خطی و بدون خط کسری می‌نویسیم:
\begin{equation*}
    a / (b + c) + d / (a + d) - (a + d) / e
\end{equation*}
سپس می‌توانیم از دانش ساختمان داده، به کمک یک
stack
عبارت postfix را بنویسیم.
\begin{equation*}
    a b c + / d a d + / + a d + e / -
\end{equation*}
در نهایت عبارت را از چپ به راست می‌خوانیم. هر جا که متغیر دیدیدم آنرا به در استک
push
می‌کنیم. وقتی که به عملگرد می‌رسیم آنرا اجرا می‌کنیم و در نهایت جواب نهایی را
pop
می‌کنیم.
\codesample{codes/1-4.asm}